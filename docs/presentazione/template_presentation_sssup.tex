\documentclass{beamer}
\usetheme{JLTree}

\usepackage{graphicx}
\usepackage[italian]{babel}
\usepackage{amsfonts}
\usepackage{url}
\usepackage{amsmath, amssymb, amsthm}
\usepackage{multicol}

\newcommand{\numberset}{\mathbb}
\newcommand{\N}{\numberset{N}}
\newcommand{\R}{\numberset{R}}
\newcommand{\Z}{\numberset{Z}}
\newcommand{\pscal}{\text{\large{\textbf{\textperiodcentered}}}}






% opening
\title[Algoritmo genetico parallelo sul Single-Chip Cloud Computer]{Algoritmo genetico parallelo sul Single-Chip Cloud Computer}
\author[\hspace{2em} Vincenzo Maffione\hspace{3.6cm} Tesi di Licenza \hspace{2.5cm} 8 Luglio 2011]{}
\date{}


\begin{document}


\begin{frame}
\vspace*{-0.1cm}
    \titlepage
\vspace*{-2.6cm}

\begin{center}
{\footnotesize Tesi di Licenza\\
Scuola Superiore Sant'Anna}
\end{center}

\begin{center}
\begin{tabular}{l p{2.5cm} r c}

\footnotesize\textsc{Relatore} & & \footnotesize\textsc{Candidato} \\
{\small Prof. \textit{Giuseppe Lipari}} & &\small\textit{Vincenzo Maffione}\\
\vspace{0.4em}
{\footnotesize Scuola Superiore Sant'Anna} & &\\
\footnotesize\textsc{Tutor} & &  \\
{\small Prof. \textit{Paolo Ancilotti}} && \\
{\footnotesize Scuola Superiore Sant'Anna} && \\
\end{tabular}
\end{center}

\begin{center}
{\small 8 Luglio 2011}
\end{center}

\end{frame}


%=================================================================
\section{Introduzione}
%=================================================================

\begin{frame}
  \frametitle{Tendenze architetturali}

  \begin{block}{Caratterizzazione}
Rappresentano tutti e soli gli operatori funzionali monotoni \alert{invarianti al contrasto}.
  \end{block}

  \begin{block}{Principio di funzionamento}
    \begin{enumerate}

      \item decomposizione dell'ingresso nell'insieme delle sue \emph{cross sections}

      \item applicazione di un'operatore \emph{binario}, \emph{crescente} ed \emph{invariante alla traslazione}

      \item ricostruzione dell'uscita tramite il principio di sovrapposizione degli effetti

    \end{enumerate}
  \end{block}
\end{frame}


\begin{frame}
  \frametitle{Schemi alle Differenze Finite (FDS) per moti di curvatura}

  Tutte le scale spaziali \emph{causali}, \emph{locali}, \emph{isometriche} ed \emph{invarianti al contrasto} si basano su EDP del tipo

$$
\frac{\partial u}{\partial t}=g({\rm curv}(u),t)|Du|
$$

\vspace{0.5em}

  \begin{description} 
    \item[Moto di Curvatura Medio (MCM):] $g({\rm curv}(u),t)={\rm curv}(u)$
  \end{description}

  \begin{block}{Tecniche di simulazione numerica}

    \begin{itemize}
      \item interpolazione di Shannon e utilizzo di operatori differenziali continui
      \item costruzione di uno schema discreto che soddisfi requisiti di \emph{consistenza}, \emph{stabilit\`a} e \emph{convergenza}
    \end{itemize}
  \end{block}

\end{frame}



%=================================================================
\section{Descrizione dell'Algoritmo}
%=================================================================

\begin{frame}

  \frametitle{Analisi del codice C - \alert{main} function}

  {\bf Input:} immagine iniziale $u_0$

  \vspace{0.5em}

  {\bf Output:} immagine $u$ evoluta ad una scala normalizzata $R$

  \vspace{0.5em}

  \begin{itemize}
    \item leggi $u_0$ e valuta il numero di canali da processare 
    \item calcola il numero di iterazioni necessarie, $n_{iter}=\displaystyle\frac{R^2}{2\Delta t}$
    \item considera l'insieme di livello associato al valore di grigio $k$-esimo e prendi $v_k(i,j)=\left\{
  \begin{array}{ll}
    255 & \hbox{ se } u(i,j) \ge k \\
    0 & \hbox{ altrimenti } \\
  \end{array}
\right.
$
    \item applica la funzione \alert{mcm} a $v_k$, ottenendo l'array $w_k$
    \item $u(i,j)=\sup\{\lambda\mbox{ }|\mbox{ } w_{\lambda}(i,j)\ge 127,5\}$
  \end{itemize}

\end{frame}


\begin{frame}

  \frametitle{Analisi del codice C - \alert{mcm} function}

  \only<1>{

    \begin{itemize}
     \item calcola le derivate parziali $u_x$ e $u_y$ ed il gradiente discreto $Du$
    \end{itemize}

    \begin{center}
    \begin{enumerate}
      \item $u_x= \displaystyle \frac{1}{8\Delta x} \cdot u_n\star
\begin{pmatrix}
-1  & 0 & 1 \\
-2  & 0 & 2 \\
-1  & 0 & 1 
\end{pmatrix}$

      \item $u_y= \displaystyle\frac{1}{8\Delta x} \cdot u_n\star
\begin{pmatrix}
-1 & -2 & -1 \\
0  &  0 &  0 \\
1  &  2 &  1 
\end{pmatrix}$
	      
      \item $Du=\displaystyle\sqrt{u_x^2+u_y^2}$

    \end{enumerate}
    \end{center}

    \begin{block}{}
     Gli schemi numerici sono \alert{consistenti} con gli operatori differenziali.
    \end{block}

}

  \only<2>{
    \begin{itemize}
      \item calcola le derivate parziali $u_x$, $u_y$ ed il gradiente discreto $Du$
      \item se $|Du|<T_g \Rightarrow $ applica l'Equazione del Calore (diffusione isotropica)
    \end{itemize}
    
    \begin{center}
    \begin{enumerate}

      \item $\Delta u_n= \displaystyle\frac{1}{\Delta x^2}\cdot u_n \star
\begin{pmatrix}
0 & 1  & 0 \\
1 & -4 & 1 \\
0 & 1  & 0 
\end{pmatrix}
$
      \item $u_{n+1}=u_n+ \displaystyle\frac{\Delta t}{2}\cdot \Delta u_n$
    \end{enumerate}
    \end{center}

    \begin{block}{}
La successione $u_n$ \`e \alert{consistente} per ogni valore di $\Delta t$ e \alert{stabile} per $\Delta t \le 0,5$. 
    \end{block}
}

\only<3>{

    \begin{itemize}
      \item calcola le derivate parziali $u_x$, $u_y$ ed il gradiente discreto $Du$
      \item se $|Du|<T_g \Rightarrow $ applica l'Equazione del Calore (diffusione isotropica)
      \item se $|Du|>T_g \Rightarrow $ applica uno schema lineare basato sulla convoluzione con un kernel variabile
    \end{itemize}

    \begin{center}
    \begin{enumerate}

      \item $(u_{\xi\xi})_n = \displaystyle\frac{1}{\Delta x^2}\cdot u_n \star \begin{pmatrix}
\lambda_3 & \lambda_2   & \lambda_4 \\
\lambda_1 & -4\lambda_0 & \lambda_1 \\
\lambda_4 & \lambda_2   & \lambda_3 \\
\end{pmatrix}$
\vspace{1em}
      \item $u_{n+1}(i,j)=u_n(i,j)+\Delta t \cdot (u_{\xi\xi})_n(i,j)$ 
    \end{enumerate}
    \end{center}

    \begin{block}{}
La scelta dei $\lambda_i$ segue criteri di \alert{stabilit\`a} e consente di soddisfare requisiti geometrici ed analitici. 
    \end{block}
}

\end{frame}

%=================================================================
\section{Testing e Scelta del Passo Temporale}
%=================================================================

\begin{frame}

  \frametitle{Invarianza al contrasto}

  \only<1>{
    \begin{block}{}
  Calcola il \emph{RMSE} al variare di $\Delta t$ dopo aver applicato stack filter + cambiamento di contrasto e viceversa.
    \end{block}
    \begin{figure}
      %\includegraphics[width=.95\columnwidth]{contrinvdifftimesteps_lin_imgbis}
    \end{figure}
 
}

  \only<2>{
    \begin{block}{}
  Calcola il \emph{RMSE} dopo aver applicato MCM + cambiamento di contrasto lineare e viceversa.
    \end{block}
    \begin{figure}
      %\includegraphics[width=.95\columnwidth]{contrinvstackmcm_lin_imgbis}
    \end{figure}
}

  \only<3>{
    \begin{block}{}
  Calcola il \emph{RMSE} dopo aver applicato MCM + cambiamento di contrasto quadratico e viceversa.
    \end{block}
   \begin{figure}
    %\includegraphics[width=.95\columnwidth]{contrinvstackmcm_pow_imgbis}
   \end{figure}
}

\end{frame}


\begin{frame}
 
  \frametitle{Propriet\`a di denoising}

  \only<1>{
    \begin{block}{}
  Contamina una figura geometrica con $p=30\%$ di rumore uniforme e applica lo stack filter, calcolando il \emph{RMSE} al variare di $\Delta t$.
    \end{block}
   \begin{figure}
    %\includegraphics[width=.95\columnwidth]{unoisedifftimestepsbis}
   \end{figure}
}

  \only<2>{
    \begin{block}{}
  Contamina una figura geometrica con $p=10\%$ di rumore uniforme e calcola il \emph{RMSE}, dopo aver applicato stack filter/FDS originale.
    \end{block}
    \begin{figure}
     %\includegraphics[width=.95\columnwidth]{unoise10percbisbis}
    \end{figure}
}

  \only<3>{
    \begin{block}{}
  Contamina una figura geometrica con $p=30\%$ di rumore uniforme e calcola il \emph{RMSE}, dopo aver applicato stack filter/FDS originale.
    \end{block}
    \begin{figure}
     %\includegraphics[width=.95\columnwidth]{unoise30percbisbis}
    \end{figure}
}

\end{frame}


\begin{frame}

  \frametitle{Tempo di processing}

  \begin{block}{Problema}
Il tempo richiesto dall'algoritmo aumenta in modo notevole quando si utilizza lo stack filter anzich\`e lo schema originale.
  \end{block}

  \begin{block}{Possibili soluzioni}
    \begin{itemize}
      \item aumentare $\Delta t$ fino all'upper bound teorico di $0,5$, dovuto all'utilizzo dell'Equazione del Calore
      \item parallelizzare opportune porzioni di codice (\emph{OpenMP})
      \item ridurre il numero di insiemi di livello processati
    \end{itemize}
  \end{block}

\end{frame}



%=================================================================
\section{Esempi e Conclusioni}
%=================================================================


\begin{frame}
 
  \frametitle{Figure geometriche - 1}

  \begin{columns}

  \column[c]{.5\textwidth}
    \begin{center}
Stack Filter\\  
\vspace*{-0.5em}
    \begin{figure}
      
    \end{figure}
    \end{center}

  \column[c]{.5\textwidth}
    \begin{center}
FDS originale\\
\vspace*{-0.5em}
    \begin{figure}
      
    \end{figure}
    \end{center}

  \end{columns}

  \begin{block}{}
   \begin{itemize}
    \item eliminazione dell'effetto di tassellizzazione
    \item lo stack filter produce contorni pi\`u netti e definiti 
   \end{itemize}
  \end{block}

\end{frame}


\begin{frame}

  \frametitle{Figure geometriche - 2}
\vspace*{-0.5em}
  \begin{columns}

  \column[c]{.5\textwidth}
    \begin{center}
Stack Filter\\  
\vspace*{-1.1em}
    \begin{figure}
      
    \end{figure}
    \end{center}

  \column[c]{.5\textwidth}
    \begin{center}
FDS originale\\  
\vspace*{-1.1em}
    \begin{figure}
      
    \end{figure}
    \end{center}

  \end{columns}

  \begin{block}{}
   \begin{itemize}
    \item lo stack filter non crea nuovi livelli di grigio
    \item un insieme convesso diventa concavo e si riduce ad un punto con profilo limitante circolare 
   \end{itemize}
  \end{block}
 
\end{frame}


\begin{frame}
 
  \frametitle{Eliminazione del rumore in un'immagine B/W ($p=30\%$)}

  \only<1>{

  \begin{columns}

  \column[c]{.5\textwidth}

    \begin{center}
Immagine originale
\vspace*{-0.5em}
    \begin{figure}
      %\includegraphics[width=.75\textwidth]{eifnew}
    \end{figure}
    \end{center}

  \column[c]{.5\textwidth}

    \begin{center}     
Immagine affetta da rumore
\vspace*{-0.5em}
    \begin{figure}
      %\includegraphics[width=.75\textwidth]{eifnew_unoise30}
    \end{figure}
    \end{center}

  \end{columns}

}

  \only<2>{


  \begin{columns}

  \column[c]{.3\textwidth}

    \begin{center}
\vspace*{-1.5em}
{\scriptsize Immagine originale}
\vspace*{-0.7em}
    \begin{figure}
      %\includegraphics[width=.5\textwidth]{eifnew}
    \end{figure}
    \end{center}

\vspace*{-0.5cm}

    \begin{center}
{\scriptsize Immagine + rumore}
\vspace*{-0.7em}
    \begin{figure}
      %\includegraphics[width=.5\textwidth]{eifnew_unoise30}
    \end{figure}
    \end{center}

  \column[c]{.45\textwidth}
  
    \begin{center}     
Stack Filter
\vspace*{-0.5em}
    \begin{figure}
%\includegraphics[width=.8\textwidth]{eifnew_unoise30_stackelimO3_step0dot5_sc2dot5}
    \end{figure}
    \end{center}

  \column[c]{.45\textwidth}
    \begin{center}
FDS originale
\vspace*{-0.5em}
    \begin{figure}
%      \includegraphics[width=.8\textwidth]{eifnew_unoise30_mcmO3_step0dot1_sc3}
    \end{figure}
    \end{center}

  \end{columns}

}

\end{frame}

\begin{frame}
 
  \frametitle{Eliminazione del rumore in un'immagine RGB ($p=30\%$)}
  \only<1>{

  \begin{columns}

  \column[c]{.5\textwidth}

    \begin{center}
Immagine originale
\vspace*{-0.5em}
    \begin{figure}
%      \includegraphics[width=.95\textwidth]{naturefence}
    \end{figure}
    \end{center}

  \column[c]{.5\textwidth}

    \begin{center}     
Immagine affetta da rumore
\vspace*{-0.5em}
    \begin{figure}
%      \includegraphics[width=.95\textwidth]{naturefence_unoise30}
    \end{figure}
    \end{center}

  \end{columns}

}

  \only<2>{


  \begin{columns}

  \column[c]{.25\textwidth}

    \begin{center}
\vspace*{-1.5em}
{\scriptsize Immagine originale}
\vspace*{-0.7em}
    \begin{figure}
%      \includegraphics[width=.85\textwidth]{naturefence}
    \end{figure}
    \end{center}

\vspace*{-0.7cm}

    \begin{center}
{\scriptsize Immagine + rumore}
\vspace*{-0.7em}
    \begin{figure}
%      \includegraphics[width=.85\textwidth]{naturefence_unoise30}
    \end{figure}
    \end{center}

  \column[c]{.45\textwidth}
  
    \begin{center}     
Stack Filter
\vspace*{-0.5em}
    \begin{figure}
      %\includegraphics[width=.95\textwidth]{naturefence_unoise30_stackelimO3_step0dot5_sc2dot5}
    \end{figure}
    \end{center}

  \column[c]{.45\textwidth}
    \begin{center}
FDS originale
\vspace*{-0.5em}
    \begin{figure}
      %\includegraphics[width=.95\textwidth]{naturefence_unoise30_mcmO3_step0dot1_sc4}
    \end{figure}
    \end{center}

  \end{columns}
  \begin{block}{}
    A differenza dello schema originario, lo stack filter elimina con successo il rumore additivo uniforme.
  \end{block}
}

\end{frame}

\begin{frame}
 
  \frametitle{Conclusioni: pro e contro}

Lo stack filter per la simulazione del Moto di Curvatura Medio svolge una funzione di \emph{smoothing} propedeutica ad un'ulteriore analisi ed estrazione del contenuto informativo delle immagini.

  \begin{columns}

  \column[c]{.5\textwidth}

    \begin{block}{Vantaggi}
     \begin{itemize}
      \item stabilit\`a
      \item invarianza al contrasto
      \item buone propriet\`a di denoising
     \end{itemize}
    \end{block}

  \column[c]{.5\textwidth}

    \begin{block}{Svantaggi}
     \begin{itemize}
      \item tempi di processing non trascurabili
      \item dipendenza da una griglia
     \end{itemize}
    \end{block}

  \end{columns}

  \begin{alertblock}{}
    A breve sul sito \url{www.ipol.im} sar\`a possibile scaricare il codice in linguaggio C e sar\`a disponibile una descrizione dettagliata dell'algoritmo testabile direttamente online.
  \end{alertblock}

\end{frame}


\end{document}
